\documentclass[a4paper,11pt]{article}
\pdfoutput=1 % if your are submitting a pdflatex (i.e. if you have
             % images in pdf, png or jpg format)

\usepackage{jheppub} % for details on the use of the package, please
                     % see the JHEP-author-manual

\usepackage[T1]{fontenc} % if needed
\usepackage{mathtools}
\usepackage{amsthm}

\def\signed #1{{\leavevmode\unskip\nobreak\hfil\penalty50\hskip2em
  \hbox{}\nobreak\hfil(#1)%
  \parfillskip=0pt \finalhyphendemerits=0 \endgraf}}

\newsavebox\mybox
\newenvironment{aquote}[1]
  {\savebox\mybox{#1}\begin{quote}}
  {\signed{\usebox\mybox}\end{quote}}

\def \diff{\mathrm{d}}
\newtheorem{theorem}{Theorem}[section]
\newtheorem{corollary}{Corollary}[theorem]
\newtheorem{lemma}[theorem]{Lemma}
\theoremstyle{definition}
\newtheorem{definition}{Definition}[section]


\title{\boldmath Quantum Mechanics}


%% %simple case: 2 authors, same institution
%% \author{A. Uthor}
%% \author{and A. Nother Author}
%% \affiliation{Institution,\\Address, Country}

% more complex case: 4 authors, 3 institutions, 2 footnotes
\author[]{Mingde Ren}

% The "\note" macro will give a warning: "Ignoring empty anchor..."
% you can safely ignore it.

\affiliation[]{South University of Science and Technology of China, Shenzhen, China}

% e-mail addresses: one for each author, in the same order as the authors
\emailAdd{11710919@mail.sustech.edu.cn}




\abstract{This is the note for a seminar on Quantum Mechanics in the spring semester, 2020. The seminar is mainly aimed for people interested in quantum mechanics with mathematical background. The lectures will focus on the algebraic structures in quantum mechanics, rather than talking about the PDEs or using methods in functional analysis, for there have already been lots of books in such topics.}



\begin{document} 
\maketitle
%\flushbottom

\section{Introduction}
Quantum mechanics is the most elegant theory that human beings have ever built. It helped us to create computers (by which I mean, classical computers), to invent new materials, or even to come up with new ideas in mathematics. The success of it is so fruitful that until now, a hundred years later, we are still making new discoveries out of it that are beyond our imagination.

Books being written, stories being told, heroes being talked about all the time. And now modern science fictions especially favour that magic word -- QUANTUM, the word that injects magic power into every seemingly trivial noun that follows. The brightness of quantum mechanics  seems to be shining enough so that it won't be harmful to ignore the dark.

However, it is not elegant enough.

If you ever come close, you shall find imperfections hiding in every corner of the theory. You could also do fairly well away from the corners. Nevertheless, those annoying dirty corners are where the astonishing secrets, if there were, prefer to live.

In a seminar lasting for less than half a year, I have no dare to say I could tell the whole story, with most of which still being a riddle for myself. Thus we shall focus only on a part of it. Then let's decide which. Since there have already been tons of books about quantum mechanics and since the seminar is for mathematicians, we should tell something different.

Instead of the calculations, we shall focus more on structures; instead of analysis, we shall focus more on algebra; and instead of the success, we shall focus more on the failure.

\subsection{Background}
Quantum mechanics was not noticed until 1905, when Planck entered the stage. But actually many optical phenomena observed before that can only be explained with quantum mechanics. What happened then has been talked about in so many books so I won't do it here.

One thing to note is that in many books quantum mechanics is treated as some opposite of the classical one. However, that treatment is not totally correct. Quantum mechanics shares a strong connection with the classical one, which is a fact but perhaps not a good one. I would like to say that maybe our quantum mechanics is not "quantum" enough so that it relies heavily on the framework of classical mechanics.

It is often said that quantum mechanics is really abstract. Maybe it is true. But a more interesting question to ask is "why isn't classical mechanics abstract?" There are three reasons for that. First of all, I think both of them are abstract when you are unfamiliar with them. Our well-established education system made us acquire lots of knowledge of Newtonian mechanics in high schools. After that, the world seems to work harmonically. Another reason is that classical mechanics is enough for most phenomena in every-day life, while it needs extreme experiment condition to see quantum effects. The last reason is that the word \textit{force} lies in the vocabulary far before classical mechanics being built. Actually the word \textit{force} can be regarded to have been redefined when we assign to it a magnitude. But in quantum mechanics, we created words and their quantitative descriptions at the same time.

However, for mathematicians, being abstract is of course never terrifying. So it won't be hard for us to proceed.

\subsection{Why is quantum mechanics so messy}
One amazing thing of quantum mechanics is that it was not built by one particular person, unlike Einstein's relativity, but by so many heroes of that time, which, however, also leads to the messiness of the theory.

\begin{aquote}{Rene Descartes}
"... that there is very often less perfection in works composed of several portions, and carried out by the hands of various masters, than in those on which one individual alone has worked. Thus we see that buildings planned and carried out by one architect alone are usually more beautiful and better proportioned than those which many have tried to put in order and improve, making use of the old walls which were built with other ends in view..."\cite{Discourse on Method}
\end{aquote}

As noticed by Descartes, it is not easy for distinct masters to tell the same story and make it coherent at the same time. After decades of struggling, now we finally arrived at a fairly coherent theory but with fundamental problems left unsolved, especially problems related to measurements in quantum mechanics.

There are around eight or nine different ways to describe quantum mechanics, with each of them being distinct from the others. Here I could only talk about the most well-established ones. It is unbelievable that those seemingly uncorrelated descriptions are telling the same story while all being correct.

Here I shall note the importance of equivalent descriptions for the same theory in physics. The most famous example, in my opinion, is classical mechanics. There are three equivalent descriptions, i.e., the Newtonian, the Lagrangian and the Hamiltonian. If a complete coherent theory is all we want, maybe the Newtonian mechanics is enough, why bothering looking for its equivalence?

However, quantum mechanics inherits the language of Hamiltonian mechanics, which itself inherits from the Lagrangian one. One interesting question to think is that if we did not have the Lagrangian mechanics nor the Hamiltonian mechanics, how should we build the quantum theory? Above all, I could claim that people would still have discovered quantum mechanics, but maybe with a totally different language as we are using today, since no matter what theory we have, the experiments were showing that the classical theory was incomplete. One fatal problem for Newtonian mechanics is that the Newtonian mechanics relies on the concept of force, which is hard to find a corresponding in quantum mechanics, while the other two description of classical mechanics do not.

So even though maybe we could use some technique to "quantize" the Newtonian mechanics, its brothers are more nature choices.

Ergo, the importance of equivalent descriptions is that they provide different languages and having more languages provides better chance to describe the unknown.

Now we should also be positive seeking more equivalent languages for quantum mechanics. Maybe the one you find would be the key to the next page.

\subsection{Different mathematical approaches to QM}
Let's talk about maths. As advertised, there are many different mathematical approaches to quantum mechanics. Among them there are mainly two types, one being analysis and the other being algebra.

The most important formula in quantum mechanics is the well-known Schrodinger equation. So we could use methods in PDE to study quantum mechanics, focusing on solving the Schrodinger equation. Maybe like Griffiths' choice, to put the Schrodinger equation in the very first page of the book as a friendly opening to the lovely freshmen planning to major in physics. And then talk about separations of variables, perturbations, etc. That a good way to organize quantum mechanics, as it taught you how to calculate. Or maybe being more mathematical, giving lots of tricky boundary conditions to see how the poor particle reacts.

Another approach will be algebraic, mainly linear algebra. I think this approach is more natural than the former one.

In between lies the functional analysis. Actually this is the most natural way. But if we choose to ignore the tricky problems caused by infinite dimensions, it reduces to linear algebra. That is why linear algebra is a good approach. I must admit there is a huge gap here by ignoring infinity but since there are already fruitful results doing so, physicists cannot help to touch them. Also, the operator algebra part of functional analysis is what is really useful in quantum mechanics and that is quite algebraic so quantum mechanics can be really algebraic.

Before we get started, let's first do a brief review of the necessary knowledge in classical mechanics.

\section{A brief story of classical mechanics}
\subsection{Historical line}
It occurs very often that the logical line reverses the historical line. The following is such a case. Writing with respect to the logical line is better for the coherence of the story while it is harder to understand. So I shall first talk a little bit about the history before getting into the topic, which is better for the understanding of the motivation for proposing those theories.

The very first person to be mentioned is definitely Newton, who published \textit{mathematical principles of natural philosophy} in 1686, which can be regarded as the establishment of Newtonian mechanics.

The second person was another natural philosopher, Euler. He elaborated the subject \textit{calculus of variations} in 1733. The first problem raised in the history of calculus of variation is \textit{brachistochrone curve}, the curve of fastest descent, proposed by Johann Bernoulli in 1696. Using the language of calculus of variation, the problem is reformulated as

\begin{center}
    Find a function $y$ that minimize the following functional:
    \begin{equation}
        T[y] = \int_{x_1}^{x_2} \sqrt{1+y'^2} \diff x.
    \end{equation}
\end{center}

The problem can be generalised to functionals of such kind: $T[y]=\int_{x_1}^{x_2}F(y,\dot{y},x)\diff x$ for some function $F$. And using calculus of variation, the problem is reverted to a differential equation called Euler-Lagrange equation

\begin{equation}
    \label{eq2.2}
    \frac{\partial F}{\partial y} - \frac{\diff}{\diff x}\frac{\partial F}{\partial \dot{y}} = 0.
\end{equation}

\textbf{REMARK}:
\begin{itemize}
    \item The notation $\frac{\partial F}{\partial \dot{y}}$ is a little bit confusing. For the function $F(a, b, c)$, it means $\frac{\partial F}{\partial b}\big \vert_{b=\dot{y}}$, since $\dot{x}$ is not really an argument of $F$.
    \item Lagrange also contributed a lot to this subject. He was working with Euler.
\end{itemize}

The third person was Lagrange. Newtonian mechanics uses force to deal with problems. However, when we need to solve systems with constraints, force might be not convenient. So people were looking for substitutes. The D'Alembert's principle is a good method to deal with constraints with forces. And Lagrange found that forces can be avoided and reformulated classical mechanics in his \textit{Mécanique analytique}, 1788, in which he used the concept of \textit{generalized coordinate} and using D'Alembert's principle he obtained the equation of motion as:

\begin{equation}
    \label{eq2.3}
    \frac{\diff}{\diff t}\frac{\partial T}{\partial \dot{x}} - \frac{\partial T}{\partial x} + \frac{\partial V}{\partial x} = 0,
\end{equation}
where $T$ represents kinetic energy and $V$ for potential energy.

Both Euler and he realised some idea of the principle of least action, but not in the modern form. This is formulated by the fourth person, Hamilton:

If you define a function called \textit{Lagrangian} as $L(x, \dot{x}, t) = T - V$, then since $\partial V/\partial \dot{x} = 0$, \eqref{eq2.3} becomes

\begin{equation}
    \label{eq2.4}
    \frac{\partial L}{\partial x} - \frac{\diff}{\diff t}\frac{\partial L}{\partial \dot{x}} = 0.
\end{equation}

You see that \eqref{eq2.4} is of the same form as \eqref{eq2.2}! Thus Hamilton proposed that the equation of motion is a condition for the minimization of some functional. The functional is now known as the \textit{action}:

\begin{equation}
    S[x]\coloneqq\int_{t_1}^{t_2}L(x, \dot{x}, t)\diff t.
\end{equation}

This is known as Hamilton principle.

In addition, Hamilton proposed another reformulation of classical mechanics, which uses the concept of both \textit{generalized coordinate} and \textit{generalized momentum}.

Mathematically, the Lagrangian mechanics is formulated with the tangent bundle of configuration manifold while the Hamiltonian one is with the cotangent bundle, which differs from the former by a Legendre transformation.

\subsection{Lagrange formulism}
With constraints, the configuration space of a classical system will be a submanifold of the  Euclidean space. In this case, it might be redundant to still use Cartesian coordinates since they are not independent when constrained to the manifold. That leads to our first definition:

\begin{definition}
    Given a classical system, if a set of independent variables can totally determine the configuration of the system, then they are called general coordinates of the system.
\end{definition}

\textbf{REMARK}:
\begin{itemize}
    \item Remark for the remarks. Things in the REMARK part are not necessarily related to the main context. It can be aimed for anyone. It is just something I want to note for the current topic. Do not get bothered if you came into some word you do not understand. But finishing reading those words is appreciated.
    \item I won't give \textit{classical system} a definition since there is nothing special to note here. However, I will use this concept to define a quantum system.
    \item That is to say, a choice of general coordinates is a choice of parameterization of the configuration manifold. The number of general coordinates is equal to the number of dimension of this manifold, which is clearly invariant under changes of variables.
    \item The configuration manifold is always an embedded manifold into an higher dimensional Euclidean space. So sometimes we still use the redundant Euclidean coordinates since it is endowed with a linear structure and a natural metric structure.
\end{itemize}

\textbf{Examples}:
\begin{itemize}
    \item Free particles in three dimensions
    
    In classical physics, particles are regarded as point particles. So the configuration space is just $\mathbb{R}^3 \times \cdots \times \mathbb{R}^3 \cong \mathbb{R}^{3n}$. A choice of general coordinates is just the Euclidean coordinates of each particle.
    
    (In fact there is no natural choice for the background coordinates of the Euclidean space due to Galileo's principle of relativity. But I won't discuss it here.)
    
    \item A stick in two dimensions
    
    The configuration of the stick can be determined by its two end points(or actually any two points on it). However, there is a constraint for the two point: a constant separation. Thus the configuration space is $2+2-1=3$ dimensional and it is diffeomorphic to $\mathbb{R}^2\times S^1$.
    
    \item An infinitely thin stick in three dimensions
    
    This is nearly the same as the above one. Note that infinitely thin means there is no rotational degree of freedom along the axis, which ensures that the configuration can be determined by its end points. The configuration space is of $3+3-1=5$ dimensions and diffeomorphic to $\mathbb{R}^3\times S^2$.
    
    \item A stick of finite thickness in three dimensions
    
    In this case, only the positions of two end points can no longer determine the configuration of the stick. We need to consider the rotation around the axis, which is a degree of freedom of dimension one. Thus the space is diffeomorphic to $\mathbb{R}^3\times S^2\times S^1$.
    
    Note that this is exactly the same for any rigid body in three dimensions since a configuration of a rigid body can be determined by any three points on it. There is another way to count the degrees of freedom: there are three constraints between each pair of the three points. Thus the dimension is of $3+3+3-3=6=3+2+1$.
    
    \item Simple pendulum in two dimensions
    
    This is a historic model. The configuration space of it is just $S^1$. The equation of motion for it is nonlinear. But if the angle is small enough, since $S^1$ is locally diffeomorphic to $\mathbb{R}$, the equation of motion can be approximated as that of a harmonic oscillator.
    
    \item Double pendulum in two dimensions
    
    A double pendulum is a single pendulum connected to another. Thus the configuration space is diffeomorphic to $S^1\times S^1\cong T^2$.
    
\end{itemize}

\subsection{Hamilton formulism}

\subsection{Concrete examples}
\subsubsection{Trivial examples}
\subsubsection{Normal modes and dispersion relation}
\subsubsection{Classical field theory}
\subsubsection{Maxwell's theory and winding number}


\section{Introduction to quantum mechanics}
\subsection{Axioms}

\subsection{Simple examples}

\section{Quantization: Where do quantum systems come from?}
\subsection{Canonical quantization}
\subsubsection{What is a quantum system?}
\subsubsection{The two Lie algebras}
\subsubsection{Why is canonical quantization "wrong"?}
\subsubsection{Examples}
\subsubsection{Bipartite systems and tensor product}
\subsubsection{How to quantize a classical field?}

\subsection{Path integral quantization}
\subsubsection{Physical Picture}
\subsubsection{From canonical quantization to path integral}
\subsubsection{Quantum theory made topological: Why TQFT?}
\subsubsection{Limitations and future outlook}

\section{Solving quantum systems}
\subsection{The Schrodinger equation}

\subsection{The history of the decline and fall of the perturbation theory}
\subsubsection{Why do we need perturbation?}
\subsubsection{When it fails}
\subsubsection{Divergent series in physics: the crazy physicists}

\subsection{Group representation theory and symmetry}


\iffalse

\section{LATEX Some examples and best-practices}
\label{sec:intro}

%%%
For internal references use label-refs: see section~\ref{sec:intro}.
Bibliographic citations can be done with cite: refs.~\cite{a,b,c}.
When possible, align equations on the equal sign. The package
\texttt{amsmath} is already loaded. See \eqref{eq:x}.
\begin{equation}
\label{eq:x}
\begin{split}
x &= 1 \,,
\qquad
y = 2 \,,
\\
z &= 3 \,.
\end{split}
\end{equation}
Also, watch out for the punctuation at the end of the equations.


If you want some equations without the tag (number), please use the available
starred-environments. For example:
\begin{equation*}
x = 1
\end{equation*}

The amsmath package has many features. For example, you can use use
\texttt{subequations} environment:
\begin{subequations}\label{eq:y}
\begin{align}
\label{eq:y:1}
a & = 1
\\
\label{eq:y:2}
b & = 2
\end{align}
and it will continue to operate across the text also.
\begin{equation}
\label{eq:y:3}
c = 3
\end{equation}
\end{subequations}
The references will work as you'd expect: \eqref{eq:y:1},
\eqref{eq:y:2} and \eqref{eq:y:3} are all part of \eqref{eq:y}.

A similar solution is available for figures via the \texttt{subfigure}
package (not loaded by default and not shown here). 
All figures and tables should be referenced in the text and should be
placed at the top of the page where they are first cited or in
subsequent pages. Positioning them in the source file
after the paragraph where you first reference them usually yield good
results. See figure~\ref{fig:i} and table~\ref{tab:i}.

\begin{figure}[tbp]
\centering % \begin{center}/\end{center} takes some additional vertical space
\includegraphics[width=.45\textwidth,trim=0 380 0 200,clip]{img1.pdf}
\hfill
\includegraphics[width=.45\textwidth,origin=c,angle=180]{img2.pdf}
% "\includegraphics" is very powerful; the graphicx package is already loaded
\caption{\label{fig:i} Always give a caption.}
\end{figure}

\begin{table}[tbp]
\centering
\begin{tabular}{|lr|c|}
\hline
x&y&x and y\\
\hline 
a & b & a and b\\
1 & 2 & 1 and 2\\
$\alpha$ & $\beta$ & $\alpha$ and $\beta$\\
\hline
\end{tabular}
\caption{\label{tab:i} We prefer to have borders around the tables.}
\end{table}

We discourage the use of inline figures (wrapfigure), as they may be
difficult to position if the page layout changes.

We suggest not to abbreviate: ``section'', ``appendix'', ``figure''
and ``table'', but ``eq.'' and ``ref.'' are welcome. Also, please do
not use \texttt{\textbackslash emph} or \texttt{\textbackslash it} for
latin abbreviaitons: i.e., et al., e.g., vs., etc.

\appendix
\section{Appendix I}
Please always give a title also for appendices.



\acknowledgments

This is the most common positions for acknowledgments. A macro is
available to maintain the same layout and spelling of the heading.

\paragraph{Note added.} This is also a good position for notes added
after the paper has been written.





% The bibliography will probably be heavily edited during typesetting.
% We'll parse it and, using the arxiv number or the journal data, will
% query inspire, trying to verify the data (this will probalby spot
% eventual typos) and retrive the document DOI and eventual errata.
% We however suggest to always provide author, title and journal data:
% in short all the informations that clearly identify a document.

% Please avoid comments such as "For a review'', "For some examples",
% "and references therein" or move them in the text. In general,
% please leave only references in the bibliography and move all
% accessory text in footnotes.

% Also, please have only one work for each \bibitem.
\fi

\begin{thebibliography}{99}

\bibitem{Discourse on Method}
Rene Descartes, \emph{Discourse on Method}.

\end{thebibliography}

\end{document}