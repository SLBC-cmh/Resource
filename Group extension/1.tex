\documentclass[UTF8]{ctexart}
\usepackage{bm}
\usepackage{amsmath}
\usepackage{amssymb}
\usepackage[utf8]{inputenc}
\usepackage[english]{babel}
\usepackage{amsthm}
\usepackage{graphicx}
\newtheorem{theorem}{Theorem}[section]
\newtheorem{corollary}{Corollary}[theorem]
\newtheorem{lemma}[theorem]{Lemma}
\theoremstyle{definition}
\newtheorem{definition}{Definition}[section]
\newtheorem{Proposition}[theorem]{Proposition}


\theoremstyle{remark}
\newtheorem*{remark}{Remark}


\title{A brief introduction to Homotopy theory and Identical Particles}
\author{11812311 岳亘}

\begin{document}
\maketitle	
	\section*{Introduction}
We have known that there are only bosons and fermions living in $\mathbb{R}^{3+1}$ from kindergarten, but what are the exact reasons? And is there something new if we pay attention to lower dimensions? In this note, I'll try to give a brief,but as coherent as possible story about identical particles in a mathematics approach.
\section*{Motivation} 
Suppose $\vert\psi\rangle$ is a state to describe $n$ identical particles, if they excahnge with each other, due to the identical property,the state should be the same only up to a phase difference.Which means, after an exchange, $\vert\psi\rangle\rightarrow e^{i\theta}\vert\psi\rangle$ .Such a process can be decribed by an element in fundamental group of identical particles configuration space.So, as long as I give a group element, there is a corresponding complex number as the phase. Then it is natural to consider the unitary representation of the fundamental group on $\mathbb{C}$.However, it is not easy to operate the fundamental group directly since the algebric strcture is not so abvious, so we try to pay attention to the braid group $B_n$ or symmetric group $S_n$, which may be isomorphic to the fundamental group.  
\section{Homotopy and Fundamental group}
First of all, we need some math background.
\begin{definition}
	Given two topological space $X,Y$,and two continuous functions $f,g:X\rightarrow Y$, a homotopy between $f,g$ is defined to be a continuous function $H:X\times [0,1]\rightarrow Y$, such that $H(x,0)=f(x),H(x,1)=g(x)$.
\end{definition}
\begin{definition}
	Given a topological space $X$, a loop based on $x\in X$ is defined to be a continuous map $\gamma:[0,1]\rightarrow X$, where $\gamma(0)=x,\gamma(1)=x$.
\end{definition} 
Two loops (functions) are called homotopy equivalent if where is a homotopy between them.
\begin{definition}
	Given a topological space $X$,choose a $x\in X$,then the fundamental group $\pi_1(X,x)$ is: $\{$loops based at $x_0$$\}$$/$(homotopy), with group multiplication:$[f_1]\cdot[f_2]:=[f_1\cdot f_2]$ ,where $$f_1\cdot f_2(t):=\left\{\begin{array}{rcl}
f_1(2t)&& 0\leq t\leq\frac{1}{2}\\f_2(2t-1)&&\frac{1}{2}\leq t\leq 1
	\end{array}\right.$$
\end{definition}
In fact, we need to show that the multiplication is well defined, which means there is a homotopy between $f_1\cdot(f_2\cdot f_3)$ and $f_1\cdot(f_2\cdot f_3)$ .We leave it as an exercise.
As some examples, it is easy to check: $\pi_1(S^n,x_o)$ is trivial for $n>1$,  $\pi_1(S^1,x_0)\simeq \mathbb{Z}$, $\pi_1(T^2,x_0)=\mathbb{Z\oplus Z}$ ($T^2$ is a torus).
\\\\Now, let's consider the $n$ identical particles configuration space. Suppose one particle lives in $\mathbb{R}^d$, considering the particles are hard-core and identical with each other, the configuation space should be $C_n(\mathbb{R}^d):=(\mathbb{R}^{nd}-\Delta)/S_n$ where $\Delta$ are the points $(\bm{r}_1,\bm{r}_2\cdots,\bm{r}_n)$ with $\bm{r}_i=\bm{r}_j$, then, by definition of the fundamental group, a exchange process is an element in $\pi_1(C_n(\mathbb{R}^d),x_0)$.
\section{Braid group and Geometric braid group}
\begin{definition}
	The $n^{th}$ (Artin) braid group $B_n$ is the group generated by   $\left\{\sigma_i\right\}$ ,where $1\leq i\leq n-1$, 
	and relations:
	$$\left\{\begin{array}{rcl}
		\sigma_i\sigma_{i+1}\sigma_i=\sigma_{i+1}\sigma_i\sigma_{i+1},&&1\leq i\leq n-2\\\sigma_i\sigma_j=\sigma_j\sigma_i,&&|i-j|\geq 2
		\end{array}\right.$$
	\end{definition}
The braid group $B_n$ is called braid group on $n$ strings, and is usually pictured using so called braid diagram.In the braid diagram, $\sigma_i$ are twists of adjacent strings, braidning $i$-th string above $i+1$-th.The inverse of each generator is braid in the other direction.The relations can be shown like the following figures:
\begin{figure}[h]
\centering
\includegraphics[width=0.8\textwidth]{11111}
\caption{Yang-Baxter relation}
\end{figure}
\begin{figure}[h]
	\centering\includegraphics[width=0.8\textwidth]{22222}
	\caption{commutation 6relation}
\end{figure}
\\In order to make braid group more visualized, we embed in into $\mathbb{R}^3$ and get the geometric braid group $\mathcal{B}_n$ 
\begin{definition}
	A geometric braid on $n$ strings $b\subset \mathbb{R}^2\times I$ is a union of $n$ disjoint topological intervals, such that:\\\\(i) each srting is mapped homeomorphically onto $I$ by projection map:\\$\mathbb{R}^2\times I\rightarrow I$\\(ii)$$\left\{\begin{array}{rcl}
	b\cap(\mathbb{R}^2\times\{0\})=\{(1,0),(2,0),\cdots,(n,0)\}\times\{0\}\\b\cap(\mathbb{R}^2\times\{1\})=\{(1,0),(2,0),\cdots,(n,0)\}\times\{1\}
	\end{array}\right.$$
\end{definition}
\begin{figure}[h]
	\centering
	\includegraphics[width=0.8\textwidth]{33333}
	\caption{A braid of three strings}
\end{figure}
\begin{definition}
	Two braids $a,b$ are called isotopic to each other, or isotopic equivalent if there is a continuous map $F$: $a\times I\rightarrow \mathbb{R}^2\times I$, such that: \\(i) $F(a,0)=a,F(b,0)=b$\\(ii) It induces a embeding map $F_s:a\rightarrow \mathbb{R}^2\times I, x\mapsto F(x,s),\forall s\in I$ which preserves the string structure, i.e.,if $x_1,x_2$ are in one string,then the image $F(x_1,s),F(x_2,s)$ are in the same corrsponding string. 
\end{definition}
\begin{Proposition}
	The braids on $n$ strings become a geometic braid group with the group elements:
	$\{$braids$\}$/(isotopy), and under the multiplication:
	$[b_1]\cdot[b_2]:=[b_1\cdot b_2]$ ,where\\ $b_1\cdot b_2:=\{(x,y,z)\in \mathbb{R}^2\times I|(x,y,2t)\in b_1, 0\leq t\leq \frac{1}{2}; (x,y,2t-1)\in b_2, \frac{1}{2}\leq t\leq 1\}$
\end{Proposition}
\begin{theorem}[Provrd by Artin]
	$\mathcal{B}_n\simeq B_n$
\end{theorem}
You can find out Artin's paper for the detials, and I will not  give a proof here.
\begin{theorem}
	$\mathcal{B}_n\simeq \pi_1(C_n(\mathbb{R}^2),x_0)$
\end{theorem}
This is the most important theorem in this note, and I will give a as detialed as possible proof.
\begin{proof}
	What we need to do is to find a group isomorphism between $\mathcal{B}_n$ and   $\pi_1(C_n(\mathbb{R}^2),x_0)$, i.e. $\phi:\mathcal{B}_n\rightarrow \pi_1(C_n(\mathbb{R}^2),x_0);b\mapsto \phi(b):I\rightarrow C_n(\mathbb{R}^2)$. There s a natural chioce to define $\phi(b)(t)$, which is using  $(\mathbb{R}^2\times \{t\})$ to sever $b$, let the intersections be the image of 
	$\phi(b)(t)$. By definition of geometric group, it is clear that such a $\phi(b)$ is a continuous loop.\\ Now we need to show that $\phi$ induces a bijection $\tilde{\phi}$ from isotopic classes of geometric braids to the homotopy classes of loops in $\pi_1(C_n(\mathbb{R}^2),x_0)$\\
	\\(i) Check $\tilde{\phi}$ is well defined.\\Let $b,b'$ be two isotopic braids, i.e.,$\exists F: a\times I\rightarrow \mathbb{R}^2\times I$ satisfy the condition of an isotopy.Then it canonically induces a homotopy between $\phi(b)$ and $\phi(b')$, which is:\\
	$H:I\times I \rightarrow C_n(\mathbb{R}^2);\ (t,s)\mapsto \phi(F(b,s))(t)$\\Let's have a check:\\
	(1)$H(t,0)=\phi(F(b,0))(t)=\phi(b)(t)$\\
	(2)$H(t,1)=\phi(F(b,1))(t)=\phi(b')(t)$\\
	(3)$H(0,s)=\phi(F(b,s))(0)=x_0$\\
	(4)$H(1,s)=\phi(F(b,s))(1)=x_0$\\
	so the homotopy $H$ is well defined!((3)(4) \\\\
	(ii)$\tilde{\phi}$ is injective.\\
	 Which means, if $\tilde{\phi}(b)$ is homotopic to $\tilde{\phi}(b')$, then $b$ is isotopic to $b'$. So, now we need to find a natural isotopy $F$ induced by a homotopy $H$.And in fact, it does.\\
The homotopy between two loops in $C_n(\mathbb{R}^2)$, is given by $n$ path-path homotopy in $\mathbb{R}^2$. If we copy $\mathbb{R}^2$ for $n$ times, there is a particle lives in each corresponding  $\mathbb{R}^2$.We label the particles by $1,2,\cdots,n$. If two loops are homotopic, then the two paths of $i$-th particle in $\mathbb{R}^2$ are homotopic, and we denote this homotopy by $H_i:I\times I\rightarrow \mathbb{R}^2$, then the natural isotopy is defined by: $F:b\times I\rightarrow \mathbb{R}^2\times I; (x,s)\mapsto (H_i(t,s),t)$\begin{remark}
	$x$ contains the information of a $\mathbb{R}^2$ coordinate, which also determines the string $i$ it is in; and a $t\in I$.Then it is proper that $i$ and $t$ appear in the image of $F$.
\end{remark}
~~Now let's check $F$ is an isotopy.\\
(1): Continuity follows from the continuity of each $H_i$ and continuity of $t$.\\
(2):\\(2.1)$F(x,0)=(H_i(t,0),t)$, so $F(b,0)=b$\\(2.2)$F(x,1)=(H_i(t,1),t)$, so $F(b,1)=b'$\\(3):By definition of $F$, $F_s$ presvers the string structure automatically.\\So the isotopy $F$ is well defined and then $\tilde{\phi}$ is  injective.\\\\(iii) $\tilde{\phi}$ is surjective.\\For and arbitary loop $\gamma$ in $C_n(\mathbb{R}^2)$, we can definine a geometric braid: $b=\cup_{t\in I} \gamma(t)\times\{t\} $ and then $\phi(b)=\gamma$.\\
\\So, $\tilde{\phi}$ is a group isomophism.\\\begin{remark} $\phi(b)\phi(b')=\gamma\cdot\gamma'=\phi(b\cdot b')$ is satisfied by the definition of $\phi$, and then:\\
	$\tilde{\phi}(b)\tilde{\phi}(b')=[\gamma][\gamma']=[\gamma\cdot\gamma']=[\phi(b\cdot b')]=\tilde{\phi}(b\cdot b')$, so first it is a group homomorphism. 
\end{remark}
\end{proof}
\begin{lemma}[state without proof]
	If $X$ is a simply connected topological space, $G$ is a discrete group acting on $X$, and $\forall x\in X$,$\exists U\subset X$,a open neighbourhood of $x$ ,s.t. $U\cap g(U)=\emptyset, \forall g\in G$, then: $\pi_1(X/G)\simeq G$.
\end{lemma}
\begin{theorem}
	$\pi_1(C_n(\mathbb{R}^m),x_0)\simeq S_n$ for $m\geq 3$
\end{theorem}
\begin{proof}
	Just use the lemma 2.4. 
\end{proof}
\section{Bosons, Fermions, and Abelian anyons}
In section 2, we have proved that in $\mathbb{R}^{2+1}$, the fundamendal group of $n$ identical particles configuration space is isomorpic to $B_n$, however, in $\mathbb{R}^{m+1},m\geq 3$, the fundamendal group of $n$ identical particles configuration space is isomorpic to $S_n$. So now, we just need to consider the unitary representation of $B_n$ and $S_n$ on $\mathbb{C}$. \\\\Now, let $\rho: G\rightarrow \mathbb{U}_1; g\mapsto e^{i\theta},\theta\in[0,2\pi)$, be the unitary representation,since both $B_n$ and $S_n$ have Yang-Baxter relation: $	\sigma_i\sigma_{i+1}\sigma_i=\sigma_{i+1}\sigma_i\sigma_{i+1}$,then it is easily to get: $\rho(\sigma_i)=\rho(\sigma_{i+1})$ ,which means there is a Characteristic complex number for a hole group.\\\\
If $G=S_n$ which has $\sigma_i^2=e$ so $\rho(\sigma_i)^2=e^{2i\theta}=1$,then $\theta=0$ or $\pi$, which just gives Bosons and Fermions! So we say, bosons' state stay completely the same when exchange their position; Fermions' state will be added a mines sign everytime exchange two Fermions.\\\\If $G=B_n$, there is no extra ralation to give a special phase factor, so $\theta$ can take any value from $[0,2\pi)$. This kind of particles are so called Alelian Anyons.\\\\We summary it as a theorem:
\begin{theorem}
	Bosons and Fermions live in $\mathbb{R}^{d+1},d\geq 3$, and Anyons live in $\mathbb{R}^{2+1}$.
\end{theorem}
\begin{remark}
	If there are degeneracy, we may need higher dimensions representation, the particles realising such representations are called non-abelian anyons, which I will not pay attention to now.
\end{remark}
\section{Why $\mathbb{R}^2$ and braid group?}
Now I will try to give a physical picture to explain what happens in $\mathbb{R}^2$. Just consider two identical particles, and suppose one rotates half loop around another,then it will create a phase factor $e^{i\theta}$. Then, consider another path whcih rotates in the opposite direction but still half a loop. The phase factor is natural to be $e^{-i\theta}$. 
\begin{figure}[h]
	\centering
	\includegraphics[width=0.8\textwidth]{44444}
	\caption{A half loop rotation of two opposite paths}
\end{figure}
\\If such a process happens in $\mathbb{R}^{3}$ or higher dimension, the two paths are totally the same since we can pull the path out of the plane and continuously change it to another path, so $e^{i\theta}=e^{-i\theta}$, which gives Bosons and Fermions.However if the process happens exactly in $\mathbb{R}^2$, we can not find a continuous way to change one path to another, so there is no reason to  restrict the relation of the phase factor.\\\\Also, this process shows that, the phase factor of Bosons and Fermions is independent with the process, so $S_n$ is enough to describe, but anyons care about different paths, correspondingly, $B_n$ also cares how the strings braid with each other. That's why $S_n$ is natural for Bosons and Fermions while $B_n$ is natural for Anyons. \\\\\section*{Reference}
(1) OSKAR WEINBERGER, 2015, Bachelor’s thesis ,KTH, Royal Institute of Technology.
\\
(2) Alberto Lerda,1992 Anyons,quantum mechanics of particles with fractional statistic.\\
(3) YS.Wu, 1984, PhysRevL Volume 52, Number 24\\
(4) G,G.Laidlaw and C,M.Dewitt, 1970, PhysRevD.3.1375\end{document}


